\documentclass[12pt]{article}
\usepackage{pmmeta}
\pmcanonicalname{BestApproximation}
\pmcreated{2013-03-22 17:31:23}
\pmmodified{2013-03-22 17:31:23}
\pmowner{asteroid}{17536}
\pmmodifier{asteroid}{17536}
\pmtitle{best approximation}
\pmrecord{4}{39915}
\pmprivacy{1}
\pmauthor{asteroid}{17536}
\pmtype{Definition}
\pmcomment{trigger rebuild}
\pmclassification{msc}{41A50}
\pmsynonym{optimal approximation}{BestApproximation}

\endmetadata

% this is the default PlanetMath preamble.  as your knowledge
% of TeX increases, you will probably want to edit this, but
% it should be fine as is for beginners.

% almost certainly you want these
\usepackage{amssymb}
\usepackage{amsmath}
\usepackage{amsfonts}

% used for TeXing text within eps files
%\usepackage{psfrag}
% need this for including graphics (\includegraphics)
%\usepackage{graphicx}
% for neatly defining theorems and propositions
%\usepackage{amsthm}
% making logically defined graphics
%%%\usepackage{xypic}

% there are many more packages, add them here as you need them

% define commands here

\begin{document}
One of the \PMlinkescapetext{central} problems in approximation theory is to determine points that minimize distances (to a given point or subset). More precisely,

{\bf Problem -} Let $X$ be a metric space and $S \subseteq X$ a subset. Given $x_0 \in X$ we want to know if there exists a point in $S$ that minimizes the distance to $x_0$, i.e. if there exists $y_0 \in S$ such that
\begin{displaymath}
d(x_0,y_0)=\inf_{y \in S}d(x_0,y)
\end{displaymath}

{\bf Definition -} A point $y_0$ that \PMlinkescapetext{satisfies} the above conditions is called a {\bf best approximation} of $x_0$ in $S$.

In general, best approximations do not exist. Thus, the problem is usually to identify classes of spaces $X$ and $S$ where the existence of best approximations can be assured.

{\bf Example :} When $S$ is compact, best approximations of a given point $x_0 \in X$ in $S$ always exist.

After one assures the existence of a best approximation, one can question about its uniqueness and how to calculate it explicitly.

{\bf Remark -} There is no reason to restrict to metric spaces. The definition of best approximation can be given for pseudo-metric spaces, semimetric spaces or any other space where a suitable notion of "distance" can be given.
%%%%%
%%%%%
\end{document}
