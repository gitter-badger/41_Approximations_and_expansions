\documentclass[12pt]{article}
\usepackage{pmmeta}
\pmcanonicalname{ConverseToTaylorsTheorem}
\pmcreated{2013-03-22 15:05:42}
\pmmodified{2013-03-22 15:05:42}
\pmowner{jirka}{4157}
\pmmodifier{jirka}{4157}
\pmtitle{converse to Taylor's theorem}
\pmrecord{6}{36823}
\pmprivacy{1}
\pmauthor{jirka}{4157}
\pmtype{Theorem}
\pmcomment{trigger rebuild}
\pmclassification{msc}{41A58}
\pmsynonym{Taylor's theorem converse}{ConverseToTaylorsTheorem}
\pmrelated{TaylorSeries}
\pmrelated{BorelLemma}

% this is the default PlanetMath preamble.  as your knowledge
% of TeX increases, you will probably want to edit this, but
% it should be fine as is for beginners.

% almost certainly you want these
\usepackage{amssymb}
\usepackage{amsmath}
\usepackage{amsfonts}

% used for TeXing text within eps files
%\usepackage{psfrag}
% need this for including graphics (\includegraphics)
%\usepackage{graphicx}
% for neatly defining theorems and propositions
\usepackage{amsthm}
% making logically defined graphics
%%%\usepackage{xypic}

% there are many more packages, add them here as you need them

% define commands here
\theoremstyle{theorem}
\newtheorem*{thm}{Theorem}
\newtheorem*{lemma}{Lemma}
\newtheorem*{conj}{Conjecture}
\newtheorem*{cor}{Corollary}
\newtheorem*{example}{Example}
\theoremstyle{definition}
\newtheorem*{defn}{Definition}
\begin{document}
Let $U \subset {\mathbb{R}}^n$ be an open set.

\begin{thm}
Let $f \colon U \to {\mathbb{R}}$ be a function such that there exists a constant $C > 0$ and an integer $k \geq 0$ such that for each $x \in U$ there is a polynomial $p_x(y)$ of \PMlinkescapetext{degree} $k$ where
\begin{equation*}
\lvert f(x+y) - p_x(y) \rvert \leq C \lvert y \rvert^{k+1}
\end{equation*}
for $y$ near 0.  Then $f \in C^k(U)$ ($f$ is $k$ \PMlinkescapetext{times} continuously differentiable) and the \PMlinkname{Taylor expansion}{TaylorSeries} of \PMlinkescapetext{order} $k$ of $f$ about any $x \in U$ is given by $p_x$.
\end{thm}

Note that when $k=0$ the hypothesis of the theorem is just that $f$ is Lipschitz in $U$ which certainly makes it continuous in $U$.

\begin{thebibliography}{9}
\bibitem{Whitney:varieties}
Steven G.\@ Krantz, Harold R.\@ Parks.
{\em \PMlinkescapetext{A Primer of Real Analytic Functions}}.
Birkh\"{a}user, Boston, 2002.
\end{thebibliography}
%%%%%
%%%%%
\end{document}
