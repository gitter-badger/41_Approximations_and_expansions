\documentclass[12pt]{article}
\usepackage{pmmeta}
\pmcanonicalname{AsymptoticBoundsForFactorial}
\pmcreated{2013-03-22 15:54:25}
\pmmodified{2013-03-22 15:54:25}
\pmowner{stevecheng}{10074}
\pmmodifier{stevecheng}{10074}
\pmtitle{asymptotic bounds for factorial}
\pmrecord{7}{37910}
\pmprivacy{1}
\pmauthor{stevecheng}{10074}
\pmtype{Result}
\pmcomment{trigger rebuild}
\pmclassification{msc}{41A60}
\pmclassification{msc}{68Q25}
\pmrelated{StirlingsApproximation}
\pmrelated{GrowthOfExponentialFunction}

\usepackage{amssymb}
\usepackage{amsmath}
\usepackage{amsfonts}
\usepackage{amsthm}
%\usepackage{enumerate}
%\usepackage{graphicx}
%\usepackage{psfrag}
%%%\usepackage{xypic}

% define commands here
\providecommand{\abs}[1]{\lvert#1\rvert}
\providecommand{\absW}[1]{\left\lvert#1\right\rvert}
\providecommand{\absB}[1]{\Bigl\lvert#1\Bigr\rvert}
\providecommand{\defnterm}[1]{\emph{#1}}

\begin{document}
Stirling's formula furnishes the following asymptotic
bound for the factorial:
\begin{align*}
n! = \sqrt{2\pi} \, n^{n+1/2} e^{-n} \, \bigl( 1+ \Theta(\tfrac 1 n) \bigr)\,, \quad n \to \infty\,.
\end{align*}

The following less precise bounds are useful for counting purposes
in computer science:

\begin{align}
n! &= o(n^n) \label{eq:nf-upper} \\
n! &= \omega(n^{\lambda n}) \qquad \text{for every $0 \leq \lambda < 1$} \label{eq:nf-lower} \\
\log n! &= \Theta(n \log n) \label{eq:log-nf}
\end{align}

($o$, $\omega$, and $\Theta$ are Landau notation.)

\begin{proof}[Proof of the upper bound of equation \eqref{eq:nf-upper}]
We must show that for every $\varepsilon > 0$, we have
\[
n! \leq \varepsilon \, n^n \quad \text{for large enough $n$.}
\]

To see this, write
\begin{align*}
n! &= \underbrace{\sqrt{2\pi} \, \frac{\sqrt{n}}{e^n} \, \bigl(1 + O(\tfrac 1 n) \bigr)} \, n^n\,,
\end{align*}
and observe that the middle quantity tends to $0$ as 
$n \to \infty$, 
so it can be made smaller than any fixed $\varepsilon$
by taking $n$ large.
\end{proof}

\begin{proof}[Proof of the lower bound of equation \eqref{eq:nf-lower}]
We are to show that for every $C > 0$, we have
\[
n! \geq C \, n^{\lambda n} \quad \text{for large enough $n$.}
\]

We write:
\begin{align*}
n! &= \underbrace{\sqrt{2\pi n} \, \bigl( 1+O(\tfrac 1 n) \bigr) 
\left(\frac{n}{e^{1/(1-\lambda)}}\right)^{(1-\lambda) n}} \, n^{\lambda n}\,,
\end{align*}
and the middle expression can be made to be $\geq C$ 
by taking $n$ to be large.
\end{proof}

\begin{proof}[Proof of bound for $\log n!$, equation \eqref{eq:log-nf}]
Showing $\log n! = O (n \log n)$
is simple:
\begin{align*}
n! &= n(n-1)\dotsm(2)(1) \leq n^n \\
\log n! &\leq n \log n
\end{align*}
for all $n \geq 1$.

To show $\log n! = \Omega (n \log n)$,
we can take equation \eqref{eq:nf-lower} with the constants $C = 1$
and $\lambda = \tfrac 1 2$.  Then
\begin{align*}
\log n! \geq \tfrac 1 2 \, n \log n 
\end{align*}
for large enough $n$.  In fact, it may be checked that $n \geq 1$ suffices.
\end{proof}

\begin{thebibliography}{3}
\bibitem{CLRS}
Thomas H. Cormen, Charles E. Leiserson, Ronald L. Rivest, Clifford Stein.
{\it Introduction to Algorithms}, second edition. MIT Press, 2001.
\bibitem{Spivak}
Michael Spivak. {\it Calculus}, third edition. Publish or Perish, 1994.
\end{thebibliography}

%%%%%
%%%%%
\end{document}
