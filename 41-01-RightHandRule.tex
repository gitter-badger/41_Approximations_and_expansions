\documentclass[12pt]{article}
\usepackage{pmmeta}
\pmcanonicalname{RightHandRule}
\pmcreated{2013-03-22 15:57:41}
\pmmodified{2013-03-22 15:57:41}
\pmowner{Wkbj79}{1863}
\pmmodifier{Wkbj79}{1863}
\pmtitle{right hand rule}
\pmrecord{15}{37975}
\pmprivacy{1}
\pmauthor{Wkbj79}{1863}
\pmtype{Theorem}
\pmcomment{trigger rebuild}
\pmclassification{msc}{41-01}
\pmclassification{msc}{28-00}
\pmclassification{msc}{26A42}
\pmrelated{LeftHandRule}
\pmrelated{MidpointRule}
\pmrelated{RiemannSum}
\pmrelated{ExampleOfEstimatingARiemannIntegral}

\usepackage{amssymb}
\usepackage{amsmath}
\usepackage{amsfonts}
\begin{document}
The \emph{right hand rule} for computing the Riemann integral $\displaystyle \int\limits_a^b f(x) \, dx$ is
\[
\int\limits_a^b f(x) \, dx = \lim_{n \to \infty} \sum_{j=1}^n f \left( a+j \left( \frac{b-a}{n} \right) \right) \left( \frac{b-a}{n} \right).
\]

If the Riemann integral is considered as a measure of area under a curve, then the expressions $\displaystyle f \left( a+j \left( \frac{b-a}{n} \right) \right)$ \PMlinkescapetext{represent} the \PMlinkescapetext{heights} of the rectangles, and $\displaystyle \frac{b-a}{n}$ is the common \PMlinkescapetext{width} of the rectangles.

The Riemann integral can be approximated by using a definite value for $n$ rather than taking a limit.  In this case, the partition is $\displaystyle \left\{ \left[ a, a+\frac{b-a}{n} \right), \dots , \left[ a+\frac{(b-a)(n-1)}{n}, b \right] \right\}$, and the function is evaluated at the \PMlinkescapetext{right} endpoints of each of these intervals.  Note that this is a special case of a \PMlinkname{right Riemann sum}{RightRiemannSum} in which the $x_j$'s are evenly spaced.
%%%%%
%%%%%
\end{document}
