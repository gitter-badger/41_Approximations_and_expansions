\documentclass[12pt]{article}
\usepackage{pmmeta}
\pmcanonicalname{ProofOfWeierstrassApproximationTheoremInRn}
\pmcreated{2013-03-22 15:40:03}
\pmmodified{2013-03-22 15:40:03}
\pmowner{rspuzio}{6075}
\pmmodifier{rspuzio}{6075}
\pmtitle{proof of Weierstrass approximation theorem in R^n}
\pmrecord{8}{37603}
\pmprivacy{1}
\pmauthor{rspuzio}{6075}
\pmtype{Proof}
\pmcomment{trigger rebuild}
\pmclassification{msc}{41A10}

\endmetadata

% this is the default PlanetMath preamble.  as your knowledge
% of TeX increases, you will probably want to edit this, but
% it should be fine as is for beginners.

% almost certainly you want these
\usepackage{amssymb}
\usepackage{amsmath}
\usepackage{amsfonts}

% used for TeXing text within eps files
%\usepackage{psfrag}
% need this for including graphics (\includegraphics)
%\usepackage{graphicx}
% for neatly defining theorems and propositions
%\usepackage{amsthm}
% making logically defined graphics
%%%\usepackage{xypic}

% there are many more packages, add them here as you need them

% define commands here
\begin{document}
To show that the Weierstrass Approximaton Theorem holds in
$\mathbb{R}^n$, we will use induction on $n$.

For the sake of simplicity, consider first the case of the cubical
region $0 \le x_i \le 1$, $1 \le i \le n$.  Suppose that $f$ is a
continuous, real valued function on this region.  Let $\epsilon$ be an
arbitrary positive constant.

Since a continuous functions on compact regions are uniformly
continuous, $f$ is uniformly continuous.  Hence, there exists an
integer $N > 0$ such that $|f(a) - f(b)| < \epsilon/2$ whenever $|a -
b| \le 1/N$ and both $a$ and $b$ lie in the cubical region.

Define $\phi \colon \mathbb{R} \to \mathbb{R}$ as follows: 
\[ \phi (x) = \left\{ \begin{matrix} 0 & x < -1 \cr 1 + x & -1
\le x \le 0 \cr 1 - x & 0 \le x \le 1 \cr 0 & x > 1\end{matrix}
\right. \]
Consider the function ${\tilde f}$ defined as follows: 
\[ {\tilde f} (x_1, \ldots x_n) = \sum_{m = 0}^N \phi(N x_1 + m)
f(m/N, x_2, \ldots x_n) \]

We shall now show that $|f(x_1, \ldots, x_n) - {\tilde f}(x_1, \ldots,
x_n)| \le \epsilon/2$ whenever $(x_1, \ldots, x_n)$ lies in the
cubical region.  By way that $\phi$ was defined, only two of the terms
in the sum defining ${\tilde f}$ will differ from zero for any
particular value of $x_1$, and hence 
\[ {\tilde f} (x_1, \ldots, x_n) = (Nx - \lfloor Nx \rfloor) f \left(
{\lfloor N x_1 \rfloor \over N}, x_2, \ldots, x_n \right) + (\lfloor
Nx \rfloor + 1 - x) f \left( {\lfloor N x_1 \rfloor + 1 \over N}, x_2,
\ldots, x_n \right), \]
so
\begin{eqnarray*}
| {\tilde f} (x_1, \ldots, x_n) - f (x_1, \ldots, x_n) | &=&
| {\tilde f} (x_1, \ldots, x_n) - \{ (Nx - \lfloor Nx \rfloor) + 
(\lfloor Nx \rfloor + 1 - x)\} f (x_1, \ldots, x_n) | \cr
&\le& (Nx - \lfloor Nx \rfloor) | f \left( {\lfloor N x_1 \rfloor \over
N}, x_2, \ldots, x_n \right) -  f (x_1, X_2 \ldots, x_n) | +
(\lfloor Nx \rfloor + 1 - x) |\left( {\lfloor N x_1 \rfloor + 1 \over N}, x_2,
\ldots, x_n \right) -  f (x_1, x_2 \ldots, x_n) | \cr
&\le& (Nx - \lfloor Nx \rfloor) {\epsilon \over 2} + 
(\lfloor Nx \rfloor + 1 - x) {\epsilon \over 2} = {\epsilon \over 2}.
\end{eqnarray*}

Next, we will use the Weierstrass approximation theorem in $n-1$
dimensions and in one dimesnsion to approximate $\tilde f$ by a
polynomial.  Since $f$ is continuous and the cubical region is
compact, $f$ must be bounded on this region.  Let $M$ be an upper
bound for the absolute value of $f$ on the cubical region.  Using the
Weierstrass approximation theorem in one dimension, we conclude that
there exists a polynoial $\breve \phi$ such that $| \breve \phi (a) -
\phi (a) | < \epsilon / (4MN)$ for all $a$ in the region.  Using the
Weierstrass approximation theorem in $n-1$ dimensions, we conclude
that there exist polynomials $p_m$, $0 \le m \le N$ such that $|p_m
(x_2, \ldots, x_n) - f (m/N, x_2,\ldots x_n) | \le {\epsilon \over
4N}$. Then one has the following inequality:
\begin{eqnarray*}
| {\breve \phi} (N x_1 + m) p_m (x_2, \ldots x_n) -
\phi (N x_1 + m) f(m/N, x_2, \ldots x_n) | &=
| {\breve \phi} (N x_1 + m) p_m (x_2, \ldots x_n) -
{\breve \phi} (N x_1 + m) f(m/N, x_2, \ldots x_n) \\
&+&
{\breve \phi} (N x_1 + m) f(m/N, x_2, \ldots x_n) -
\phi (N x_1 + m) f(m/N, x_2, \ldots x_n) | \\
&\le& |{\breve \phi} (N x_1 + m)|
| p_m (x_2, \ldots x_n) -  f(m/N, x_2, \ldots x_n)| \\
&+&
| f(m/N, x_2, \ldots x_n)|
| {\breve \phi} (N x_1 + m) - \phi (N x_1 + m)| \\
&\le& 
{\epsilon \over 4N} + M {\epsilon \over 4MN} = {\epsilon \over 2N} \\ 
\end{eqnarray*}
 Define 
\[ {\breve f} (x_1, \ldots x_n) = \sum_{m = 0}^N {\breve \phi}
(N x_1 + m) p_m (x_2, \ldots x_n). \]
As a finite sum of products of polynomials, this is a polynomial.
From the above inequality, we conclude that $|{\breve f} (a) -
{\tilde f} (a)| \le \epsilon / 2$, hence $|f(a) - {\breve f} (a)| 
\le \epsilon$.

It is a simple matter of rescaling variables to conclude the
Weirestrass approximation theorem for arbitrary parallelopipeds.  Any
compact subset of $\mathbb{R}^n$ can be embedded in some paralleloped
and any continuous function on the compact subset can be extended to a
continuous function on the parallelopiped.  By approximating this
extended function, we conclude the Weierstrass approximation theorem
for arbitrary compact subsets of $\mathbb{R}^n$.
%%%%%
%%%%%
\end{document}
