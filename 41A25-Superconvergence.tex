\documentclass[12pt]{article}
\usepackage{pmmeta}
\pmcanonicalname{Superconvergence}
\pmcreated{2013-03-22 11:58:12}
\pmmodified{2013-03-22 11:58:12}
\pmowner{mathcam}{2727}
\pmmodifier{mathcam}{2727}
\pmtitle{superconvergence}
\pmrecord{15}{30793}
\pmprivacy{1}
\pmauthor{mathcam}{2727}
\pmtype{Definition}
\pmcomment{trigger rebuild}
\pmclassification{msc}{41A25}
\pmsynonym{superconverge}{Superconvergence}
%\pmkeywords{convergence}
%\pmkeywords{Newton's method}
%\pmkeywords{Kantorovitch's theorem}
\pmrelated{NewtonsMethod}
\pmrelated{KantorovitchsTheorem}
\pmrelated{SuperincreasingSequence}

\usepackage{amssymb}
\usepackage{amsmath}
\usepackage{amsfonts}
\usepackage{graphicx}
%%%\usepackage{xypic}
\begin{document}
A sequence $x_0,x_1,\dots$ \emph{superconverges to 0} if, when the $x_i$ are written in base 2, then each number $x_i$ starts with $2^i-1\approx 2^i$ zeroes.
For example, the following sequence is superconverging to 0.
$$\begin{array}{clcl}
x_{n+1}&=x_n^2 & (x_n)_{10} & (x_n)_2\\[1.5pt]
x_0 &= & \frac{1}{2} & .1\\[1.5pt]
x_1 &= & \frac{1}{4} & .01\\[1.5pt]
x_2 &= & \frac{1}{16} & .0001\\[1.5pt]
x_3 &= & \frac{1}{256} & .00000001\\[1.5pt]
x_4 &= & \frac{1}{65536} & .0000000000000001
\end{array}$$
In this case it is easy to see that the number of binary 0's doubles each $x_n$.

A sequence $\{x_i\}$ \emph{superconverges to $x$} if $\{x_i-x\}$ superconverges  to 0, and a sequence $\{y_i\}$ is said to be \emph{superconvergent} if there exists a $y$ to which the sequence superconverges.
%%%%%
%%%%%
%%%%%
\end{document}
