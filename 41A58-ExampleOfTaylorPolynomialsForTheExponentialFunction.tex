\documentclass[12pt]{article}
\usepackage{pmmeta}
\pmcanonicalname{ExampleOfTaylorPolynomialsForTheExponentialFunction}
\pmcreated{2013-03-22 15:04:09}
\pmmodified{2013-03-22 15:04:09}
\pmowner{alozano}{2414}
\pmmodifier{alozano}{2414}
\pmtitle{example of Taylor polynomials for the exponential function}
\pmrecord{6}{36790}
\pmprivacy{1}
\pmauthor{alozano}{2414}
\pmtype{Example}
\pmcomment{trigger rebuild}
\pmclassification{msc}{41A58}
%\pmkeywords{approximations of e}
\pmrelated{LogarithmFunction}
\pmrelated{NaturalLogBase}
\pmrelated{EIsTranscendental}
\pmrelated{ExponentialFunction}

% this is the default PlanetMath preamble.  as your knowledge
% of TeX increases, you will probably want to edit this, but
% it should be fine as is for beginners.

% almost certainly you want these
\usepackage{amssymb}
\usepackage{amsmath}
\usepackage{amsthm}
\usepackage{amsfonts}

% used for TeXing text within eps files
%\usepackage{psfrag}
% need this for including graphics (\includegraphics)
\usepackage{graphicx}
% for neatly defining theorems and propositions
%\usepackage{amsthm}
% making logically defined graphics
%%%\usepackage{xypic}

% there are many more packages, add them here as you need them

% define commands here

\newtheorem{thm}{Theorem}
\newtheorem{defn}{Definition}
\newtheorem{prop}{Proposition}
\newtheorem{lemma}{Lemma}
\newtheorem{cor}{Corollary}
\newtheorem{exa}{Example}

% Some sets
\newcommand{\Nats}{\mathbb{N}}
\newcommand{\Ints}{\mathbb{Z}}
\newcommand{\Reals}{\mathbb{R}}
\newcommand{\Complex}{\mathbb{C}}
\newcommand{\Rats}{\mathbb{Q}}
\begin{document}
\begin{exa}
We construct the $n$th Taylor polynomial for $f(x)=e^x$ around $x=0$. As we know all derivatives of $e^x$ equal $e^x$ and also, $e^0=1$. Therefore, $f^{(n)}(0)=1$ for any $n$. Thus:
\begin{eqnarray*}
T_1(x) &=& 1+x\\
T_2(x) &=& 1+x + \frac{x^2}{2}\\
T_3(x) &=& 1+x + \frac{x^2}{2}+ \frac{x^3}{3!}=1+x + \frac{x^2}{2} +\frac{x^3}{6}\\
T_4(x) &=& 1+x + \frac{x^2}{2}+ \frac{x^3}{3!}+ \frac{x^4}{4!}=1+x + \frac{x^2}{2}+ \frac{x^3}{6}+ \frac{x^4}{24}
\end{eqnarray*}
In fact:
$$T_n(x)=1+x + \frac{x^2}{2}+ \frac{x^3}{3!}+ \frac{x^4}{4!}+\ldots+\frac{x^n}{n!}$$
\begin{center}
\includegraphics[scale=0.7]{expon}

Comparison of $e^x$ with the Taylor pol. of deg. $1$ (green), $2$ (blue) and $3$ (pink).
\end{center}
Let us use several Taylor polynomials to find approximations of the number $e$:
\begin{eqnarray*}
e &=& 2.718281828459045\ldots\\
e\approx T_1(1) &=& 1+1=2\\
e \approx T_2(1) &=& 1+1+1/2=2.5 \\
e \approx T_3(1) &=& 1+1+1/2+1/6=8/3=2.666\bar{6} \\
e \approx T_4(1) &=& 1+1+1/2+1/6+1/24=65/24=2.708333\bar{3} \\
e \approx T_5(1) &=& 1+1+1/2+1/6+1/24+1/120=163/60=2.71666\bar{6} \\
\end{eqnarray*}
\end{exa}
%%%%%
%%%%%
\end{document}
