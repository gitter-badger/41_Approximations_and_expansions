\documentclass[12pt]{article}
\usepackage{pmmeta}
\pmcanonicalname{ExampleOfTaylorPolynomialsForsinX}
\pmcreated{2013-03-22 15:03:43}
\pmmodified{2013-03-22 15:03:43}
\pmowner{alozano}{2414}
\pmmodifier{alozano}{2414}
\pmtitle{example of Taylor polynomials for $\sin x$}
\pmrecord{7}{36782}
\pmprivacy{1}
\pmauthor{alozano}{2414}
\pmtype{Example}
\pmcomment{trigger rebuild}
\pmclassification{msc}{41A58}
%\pmkeywords{Taylor polynomial}
%\pmkeywords{pretty graphs}
%\pmkeywords{sine function}
\pmrelated{ComplexSineAndCosine}
\pmrelated{HigherOrderDerivativesOfSineAndCosine}

\endmetadata

% this is the default PlanetMath preamble.  as your knowledge
% of TeX increases, you will probably want to edit this, but
% it should be fine as is for beginners.

% almost certainly you want these
\usepackage{amssymb}
\usepackage{amsmath}
\usepackage{amsthm}
\usepackage{amsfonts}

% used for TeXing text within eps files
%\usepackage{psfrag}
% need this for including graphics (\includegraphics)
\usepackage{graphicx}
% for neatly defining theorems and propositions
%\usepackage{amsthm}
% making logically defined graphics
%%%\usepackage{xypic}

% there are many more packages, add them here as you need them

% define commands here

\newtheorem{thm}{Theorem}
\newtheorem{defn}{Definition}
\newtheorem{prop}{Proposition}
\newtheorem{lemma}{Lemma}
\newtheorem{cor}{Corollary}

% Some sets
\newcommand{\Nats}{\mathbb{N}}
\newcommand{\Ints}{\mathbb{Z}}
\newcommand{\Reals}{\mathbb{R}}
\newcommand{\Complex}{\mathbb{C}}
\newcommand{\Rats}{\mathbb{Q}}
\begin{document}
In this entry we compute several Taylor polynomials for the function $\sin x$ around $x=0$ and we produce graphs to compare the function with the corresponding Taylor polynomial. Recall that for a given function $y=f(x)$ (here we suppose $f$ is infinitely differentiable) and a point $x=a$, the Taylor polynomial of degree $n$ ($n\geq 0$) is given by:

$$T_n(x)=f(a)+f'(a)(x-a)+\frac{f''(a)}{2!}(x-a)^2+\ldots+\frac{f^{(n)}(a)}{n!}(x-a)^n$$
where $f^{(n)}$ denotes the $n$th derivative of $f(x)$. \\

From now on we assume $f(x)=\sin x$ and $a=0$. Notice that the derivatives of $\sin x$ are cyclic:

$$f'(x)=\cos x,\quad f''(x)=-\sin x, \quad f'''(x)=-\cos x, \quad f^{(4)}(x)=\sin x = f(x).$$

Therefore, the Taylor polynomials are easy to compute. In fact:
$$f^{(2n)}(0)=0, \quad f^{(2n+1)}(0)=(-1)^n$$

Thus, the first Taylor polynomial is given by:

$$T_1(x)= 0 + 1\cdot x = x$$
In the following graph one can compare the function $T_1(x)=x$ and $\sin x$.
\begin{center}
\includegraphics[scale=0.7]{taylor1}

The function $y=\sin x$ and the first Taylor polynomial.
\end{center}

Notice that $T_2(x)=T_1(x)$. More generally, $T_{2n}(x)=T_{2n-1}(x)$ so we will not compute any other even order Taylor polynomials. However, the third degree Taylor polynomial is given by the formula:

$$T_3(x)=x-\frac{x^3}{3!}=x - \frac{x^3}{6}$$

\begin{center}
\includegraphics[scale=0.7]{taylor3}

The function $y=\sin x$ and the third Taylor polynomial.
\end{center}

The Taylor polynomial of degree $5$ is given by:

$$T_5(x)=x-\frac{x^3}{3!}+\frac{x^5}{5!}$$

\begin{center}
\includegraphics[scale=0.7]{taylor5}

The function $y=\sin x$ and the fifth Taylor polynomial.
\end{center}

Next, we compute some Taylor polynomials of higher degree. In particular, the Taylor polynomial of degree $15$ has the form:

$$T_{15}(x)=x-\frac{x^3}{6} + \frac{x^5}{120} - \frac{x^7}{5040} + \frac{x^9}{362880} - \frac{x^{11}}{39916800} + \frac{x^{13}}{6227020800} - \frac{x^{15}}{1307674368000}$$

\begin{center}
\includegraphics[scale=0.7]{taylor15}

The function $y=\sin x$ and the Taylor polynomial of degree $15$.
\end{center}

Finally, we produce a detailed view of the Taylor polynomial of degree $99$. In particular, notice that the graphs are very close until $x=34$ or so, but after that $T_{99}(x)$ behaves rather jittery and wildly.

\begin{center}
\includegraphics[scale=0.7]{detailtaylor100}

A detail of the Taylor polynomial of degree $99$ (the interval $(34,39)$).
\end{center}
%%%%%
%%%%%
\end{document}
