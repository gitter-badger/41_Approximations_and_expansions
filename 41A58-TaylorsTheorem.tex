\documentclass[12pt]{article}
\usepackage{pmmeta}
\pmcanonicalname{TaylorsTheorem}
\pmcreated{2013-03-22 11:56:53}
\pmmodified{2013-03-22 11:56:53}
\pmowner{Andrea Ambrosio}{7332}
\pmmodifier{Andrea Ambrosio}{7332}
\pmtitle{Taylor's theorem}
\pmrecord{11}{30706}
\pmprivacy{1}
\pmauthor{Andrea Ambrosio}{7332}
\pmtype{Theorem}
\pmcomment{trigger rebuild}
\pmclassification{msc}{41A58}
\pmrelated{TaylorSeries}

\endmetadata

\usepackage{amssymb}
\usepackage{amsmath}
\usepackage{amsfonts}
\usepackage{graphicx}
%%%\usepackage{xypic}
\begin{document}
\section{Taylor's Theorem}

Let $f$ be a function which is defined on the interval $(a,b)$ and suppose the $n$th derivative $f^{(n)}$ exists on $(a,b)$.  Then for all $x$ and $x_0$ in $(a,b)$,

$$ R_n(x) = \frac{f^{(n)}(y)}{n!}(x-x_0)^n $$

with $y$ strictly between $x$ and $x_0$ ($y$ depends on the choice of $x$).  $R_n(x)$ is the $n$th remainder of the Taylor series for $f(x)$.
%%%%%
%%%%%
%%%%%
\end{document}
