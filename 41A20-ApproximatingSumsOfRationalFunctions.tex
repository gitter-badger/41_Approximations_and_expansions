\documentclass[12pt]{article}
\usepackage{pmmeta}
\pmcanonicalname{ApproximatingSumsOfRationalFunctions}
\pmcreated{2013-03-22 18:42:23}
\pmmodified{2013-03-22 18:42:23}
\pmowner{rspuzio}{6075}
\pmmodifier{rspuzio}{6075}
\pmtitle{approximating sums of rational functions}
\pmrecord{7}{41470}
\pmprivacy{1}
\pmauthor{rspuzio}{6075}
\pmtype{Topic}
\pmcomment{trigger rebuild}
\pmclassification{msc}{41A20}

% this is the default PlanetMath preamble.  as your knowledge
% of TeX increases, you will probably want to edit this, but
% it should be fine as is for beginners.

% almost certainly you want these
\usepackage{amssymb}
\usepackage{amsmath}
\usepackage{amsfonts}

% used for TeXing text within eps files
%\usepackage{psfrag}
% need this for including graphics (\includegraphics)
%\usepackage{graphicx}
% for neatly defining theorems and propositions
%\usepackage{amsthm}
% making logically defined graphics
%%%\usepackage{xypic}

% there are many more packages, add them here as you need them

% define commands here

\begin{document}
Given a sum of the form $\sum_{m=n}^\infty f(m)$ where $f$ is a rational
function, it is possible to approximate it by approximating $f$ by another
rational function which can be summed in closed form.  Furthermore, the
approximation so obtained becomes better as $n$ increases.

We begin with a simple illustrative example.  Suppose that we want to sum
$\sum_{m=n}^\infty 1/m^2$.  We approximate $m^2$ by $m^2 - 1/4$, which 
factors as $(m+1/2)(m-1/2)$.  Then, upon separating the approximate summand
into partial fractions, the sum collapses:
\begin{align*}
 \sum_{m=n}^\infty {1 \over (m+1/2)(m-1/2)} 
   &= \sum_{m=n}^\infty \left( {1 \over m-1/2} - {1 \over m+1/2} \right) \\
   &= \sum_{m=n}^\infty {1 \over m-1/2} -
      \sum_{m=n+1}^\infty {1 \over m-1/2} \\
   &= {1 \over n-1/2}
\end{align*}

Using a similar approach, we may estimate the error of our approximation.

[general method to come]
%%%%%
%%%%%
\end{document}
