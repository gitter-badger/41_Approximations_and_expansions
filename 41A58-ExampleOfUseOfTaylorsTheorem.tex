\documentclass[12pt]{article}
\usepackage{pmmeta}
\pmcanonicalname{ExampleOfUseOfTaylorsTheorem}
\pmcreated{2013-03-22 15:05:51}
\pmmodified{2013-03-22 15:05:51}
\pmowner{alozano}{2414}
\pmmodifier{alozano}{2414}
\pmtitle{example of use of Taylor's theorem}
\pmrecord{4}{36826}
\pmprivacy{1}
\pmauthor{alozano}{2414}
\pmtype{Example}
\pmcomment{trigger rebuild}
\pmclassification{msc}{41A58}

% this is the default PlanetMath preamble.  as your knowledge
% of TeX increases, you will probably want to edit this, but
% it should be fine as is for beginners.

% almost certainly you want these
\usepackage{amssymb}
\usepackage{amsmath}
\usepackage{amsthm}
\usepackage{amsfonts}

% used for TeXing text within eps files
%\usepackage{psfrag}
% need this for including graphics (\includegraphics)
%\usepackage{graphicx}
% for neatly defining theorems and propositions
%\usepackage{amsthm}
% making logically defined graphics
%%%\usepackage{xypic}

% there are many more packages, add them here as you need them

% define commands here

\newtheorem{thm}{Theorem}
\newtheorem{defn}{Definition}
\newtheorem{prop}{Proposition}
\newtheorem{lemma}{Lemma}
\newtheorem{cor}{Corollary}

\theoremstyle{definition}
\newtheorem{exa}[thm]{Example}

% Some sets
\newcommand{\Nats}{\mathbb{N}}
\newcommand{\Ints}{\mathbb{Z}}
\newcommand{\Reals}{\mathbb{R}}
\newcommand{\Complex}{\mathbb{C}}
\newcommand{\Rats}{\mathbb{Q}}

\newcommand{\peri}{\operatorname{Perimeter}}
\newcommand{\lc}{\lim_{x\to c}}
\newcommand{\lzero}{\lim_{x\to 0}}
\newcommand{\lhzero}{\lim_{h\to 0}}
\newcommand{\linf}{\lim_{x\to \infty}}
\newcommand{\limn}{\lim_{n\to\infty}}
\newcommand{\sumi}{\sum_{i=1}^\infty }
\newcommand{\sumn}{\sum_{n=1}^\infty }
\newcommand{\sumno}{\sum_{n=0}^\infty }
\newcommand{\sumio}{\sum_{i=1}^\infty }
\begin{document}
In this entry we use Taylor's Theorem in the following form:

\begin{thm}[Taylor's Theorem: Bounding the Error] Suppose $f$ and all its derivatives are continuous. If $T_n(x)$ is the $n$-th Taylor polynomial of $f(x)$ around $x=a$, then the error, or the difference between the real value of $f$ and the values of $T_n(x)$ is given by:
$$|E_n(x)|=|f(x)-T_n(x)|\leq \frac{M}{(n+1)!}|x-a|^{n+1}$$
where $M$ is the maximum value of $f^{(n+1)}$ (the $n+1$-th derivative of $f$) in the interval between $a$ and $x$.
\end{thm}

\begin{exa}
Suppose we want to approximate $e$ using the Taylor polynomial of degree 4, $T_4(x)$, around $x=0$ for the function $e^x$. We know that 
$$T_4(x)=1+x+x^2/2+x^3/3!+x^4/4!$$
so we are asking how close are $e$ and $T_4(1)=1+1+1/2+1/6+1/24$. In order to use the formula in the theorem, we just need to find $M$, the maximum value of the $4$th derivative of $e^x$ between $a=0$ and $x=1$. Since $f^{(4)}=e^x$ and $e^x$ is strictly increasing, the maximum in $(0,1)$ happens at $x=1$. Thus $M=e$ which is a number, say, less than $3$. Therefore:
$$|E_4(1)|=|f(1)-T_4(1)|=|e-(1+1+1/2+1/6+1/24)|\leq \frac{M}{5!}|1-0|^5=\frac{M}{5!}<\frac{3}{5!}=0.025$$
Thus the approximation has an error of less than $0.025$. Actually, if we use a calculator we obtain that the error is {\it exactly} $0.0099$. But, of course, {\it the whole point of the theorem is not to use a calculator}.
\end{exa}

\begin{exa}
What Taylor polynomial $T_n(x)$ (what $n$) should we use to approximate $e$ within $0.0001$? As above, we will be using the Taylor polynomial $T_n(x)$ for $e^x$, evaluated at $x=1$. Thus, we want the error $|E_n(1)|<0.0001$. Notice all derivatives are $e^x$ and the maximum happens at $x=1$, where $e^1=e$, so for all derivatives $M<3$. Hence by the theorem: 
$$|E_n(1)|=|f(1)-T_n(1)|=|e-(1+1+1/2+\ldots+1/n!)|\leq \frac{M}{(n+1)!}|1-0|^{n+1}=\frac{M}{(n+1)!}<\frac{3}{(n+1)!}$$
So we need $3/(n+1)! < 0.0001$. Try several values of $n$ until that is satisfied:
$$ 3/2=1.5,\ 3/3!=0.5,\ 3/4!=0.125,\ 3/5!=0.025,\ 3/6! =0.00416$$
$$ 3/7!=0.00059, \quad 3/8!=0.00007$$
Thus $n=7$ should work. So we just need $T_7(x)$, or add $1+1+1/2+\ldots+1/7!$.
\end{exa}
%%%%%
%%%%%
\end{document}
