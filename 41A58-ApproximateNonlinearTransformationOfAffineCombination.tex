\documentclass[12pt]{article}
\usepackage{pmmeta}
\pmcanonicalname{ApproximateNonlinearTransformationOfAffineCombination}
\pmcreated{2013-03-22 16:50:35}
\pmmodified{2013-03-22 16:50:35}
\pmowner{stevecheng}{10074}
\pmmodifier{stevecheng}{10074}
\pmtitle{approximate non-linear transformation of affine combination}
\pmrecord{7}{39088}
\pmprivacy{1}
\pmauthor{stevecheng}{10074}
\pmtype{Example}
\pmcomment{trigger rebuild}
\pmclassification{msc}{41A58}
\pmclassification{msc}{51N20}
%\pmkeywords{affine combination}

% The standard font packages
\usepackage{amssymb}
\usepackage{amsmath}
\usepackage{amsfonts}

% For neatly defining theorems and definitions
%\usepackage{amsthm}

% Including EPS/PDF graphics (\includegraphics)
\usepackage{graphicx}

% Making matrix-based graphics
%%%\usepackage{xypic}

% Enumeration lists with different styles
%\usepackage{enumerate}

% Set up the theorem environments
%\newtheorem{thm}{Theorem}
%\newtheorem*{thm*}{Theorem}

\providecommand{\defnterm}[1]{\emph{#1}}

% The standard number systems
\newcommand{\complex}{\mathbb{C}}
\newcommand{\real}{\mathbb{R}}
\newcommand{\rat}{\mathbb{Q}}
\newcommand{\nat}{\mathbb{N}}
\newcommand{\intset}{\mathbb{Z}}

% Absolute values and norms
% Normal, wide, and big versions of the delimeters
\providecommand{\abs}[1]{\lvert#1\rvert}
\providecommand{\absW}[1]{\left\lvert#1\right\rvert}
\providecommand{\absB}[1]{\Bigl\lvert#1\Bigr\rvert}
\providecommand{\norm}[1]{\lVert#1\rVert}
\providecommand{\normW}[1]{\left\lVert#1\right\rVert}
\providecommand{\normB}[1]{\Bigl\lVert#1\Bigr\rVert}

% Differentiation operators
\providecommand{\od}[2]{\frac{d #1}{d #2}}
\providecommand{\pd}[2]{\frac{\partial #1}{\partial #2}}
\providecommand{\pdd}[2]{\frac{\partial^2 #1}{\partial #2}}
\providecommand{\ipd}[2]{\partial #1 / \partial #2}

% Differentials on integrals
\newcommand{\dx}{\, dx}
\newcommand{\dt}{\, dt}
\newcommand{\dmu}{\, d\mu}

% Inner products
\providecommand{\ip}[2]{\langle {#1}, {#2} \rangle}

% Calligraphic letters
\newcommand{\sF}{\mathcal{F}}
\newcommand{\sD}{\mathcal{D}}

% Standard spaces
\newcommand{\Hilb}{\mathcal{H}}
\newcommand{\Le}{\mathbf{L}}

% Operators and functions occassionally used in my articles
\DeclareMathOperator{\D}{D}
\DeclareMathOperator{\linspan}{span}
\DeclareMathOperator{\rank}{rank}
\DeclareMathOperator{\lindim}{dim}
\DeclareMathOperator{\sinc}{sinc}

% Probability stuff
\newcommand{\PP}{\mathbb{P}}
\newcommand{\E}{\mathbb{E}}

\begin{document}
\PMlinkescapeword{weights}
\PMlinkescapeword{order}
\PMlinkescapeword{formula}
\PMlinkescapeword{term}
\PMlinkescapeword{terms}
\PMlinkescapeword{objects}


Consider 
applying an arbitrary transformation $f$ 
to an affine combination $\sum_{i=1}^n w_i x_i$
of some points $x_i$,
with non-negative weights $w_i$ that sum to unity.
Obviously, in general, 
\[
f \Bigl( \sum_{i=1}^n w_i x_i \Bigr) 
\neq \sum_{i=1}^n w_i f(x_i)\,.
\]
However, sometimes it is desirable to compute the image
$f(\sum_i w_i x_i)$
as if $f$ were a linear (or affine) transformation; 
the result is then hoped to be a good approximation
to the true value.
(See below for an application.)

Actually, it is possible to show that, provided that $f$ is twice 
continuously differentiable, the approximation is good to first order,
despite the absence of any derivatives of $f$ in the formula.
The domain and range of $f$ may be any normed vector spaces.

First, we write:
\begin{align*}
f\Bigl( \sum_i w_i x_i \Bigr) 
&= f\Bigl( \sum_i w_i (x_i - x_1) + x_1 \Bigr) \\
&= f(x_1) + \sum_i w_i f'(x_1) (x_i - x_1) + O\Bigl( \normB{ \sum_i w_i (x_i - x_1)}^2 \Bigr)\,.
\end{align*}

If $h = \max_i \norm{x_i - x_1}$,
then $\norm{ \sum_i w_i (x_i - x_1) } \leq \sum_i \abs{w_i} h = h$,
and so the error term in the Taylor expansion
can be simplified to $O(h^2)$.

Substituting another Taylor expansion
\begin{align*}
f(x_i) - f(x_1) = f'(x_1) (x_i - x_1) + O(h^2)\,,
\end{align*}
into the first, we obtain:
\begin{align*}
f\Bigl( \sum_i w_i x_i \Bigr)
&= f(x_1) + \sum_i w_i \Bigl( f(x_i) - f(x_1) + O(h^2) \Bigr) + O(h^2) \\
&= \sum_i w_i f(x_i) + O(h^2)\,.
\end{align*}

Furthermore, it is not hard to see,
by accounting the error from the Taylor expansions more carefully,
that we have the bound:
\[
\normB{ f\Bigl( \sum_i w_i x_i \Bigr) - \sum_i w_i f(x_i) } \leq
M h^2\,,
\]
where $M$ is the maximum,
as $\xi$ ranges inside the convex hull
formed by the points $x_i$,
of the quantity $\norm{f''(\xi)} = 
\sup\limits_{u \neq 0} \frac{\norm{f''(\xi)(u,u)}}{\norm{u}^2}$.
Finally, the point $x_1$ over which we performed Taylor expansions
can be replaced by any other point $x_k$,
and so correspondingly $h$ can be replaced by $\min_k \max_i \norm{x_i - x_k}$.

\subsection*{Application in computer graphics}

The principle just derived is often applied in vector-based computer graphics
when curved objects are drawn by cubic B\'ezier curves:
\[
\gamma(t) = \sum_{i=0}^3 w_i(t) \, x_i\,, \quad w_i(t) = \binom{3}{i} (1-t)^{n-i} t^i\,,
\]
which are affine combinations of the control points $x_i$.
To compute and display a smooth transformation $f$ of such curves,
it may be too much work to compute $f(\gamma(t))$ repeatedly
for many parameter values $t$.
Provided $\gamma$ is not too wavy,
computing and displaying $\sum_{i=0}^3 w_i(t) \, f(x_i)$
is vastly more efficient, and may result in little or 
no visually perceptible difference.

As a concrete example, consider bending a straight line segment
into a circle.
Mathematically, we are mapping the interval $[0,2\pi]$
via $t \mapsto (r\cos t, r\sin t)$.
If the interval is split into sub-segments,
each considered as a 
cubic B\'ezier curve with its interior control points
both set at the midpoint of the line segment,
then a circle can be approximated
by transforming these control points.
The following diagram shows the approximation for 24 segments (three B\'ezier
curves per $45^\circ$ arc). 

\begin{figure}[!htb]
\begin{center}
\includegraphics{circle.eps}
\end{center}
\caption{Circle drawn using approximate mapping of line segment}
\end{figure}

%%%%%
%%%%%
\end{document}
