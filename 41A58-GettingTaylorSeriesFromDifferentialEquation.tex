\documentclass[12pt]{article}
\usepackage{pmmeta}
\pmcanonicalname{GettingTaylorSeriesFromDifferentialEquation}
\pmcreated{2013-03-22 15:06:07}
\pmmodified{2013-03-22 15:06:07}
\pmowner{Wkbj79}{1863}
\pmmodifier{Wkbj79}{1863}
\pmtitle{getting Taylor series from differential equation}
\pmrecord{17}{36831}
\pmprivacy{1}
\pmauthor{Wkbj79}{1863}
\pmtype{Example}
\pmcomment{trigger rebuild}
\pmclassification{msc}{41A58}
\pmrelated{ExamplesOnHowToFindTaylorSeriesFromOtherKnownSeries}
\pmrelated{SawBladeFunction}
\pmrelated{SpecialCasesOfHypergeometricFunction}

\endmetadata

% this is the default PlanetMath preamble.  as your knowledge
% of TeX increases, you will probably want to edit this, but
% it should be fine as is for beginners.

% almost certainly you want these
\usepackage{amssymb}
\usepackage{amsmath}
\usepackage{amsfonts}

% used for TeXing text within eps files
%\usepackage{psfrag}
% need this for including graphics (\includegraphics)
%\usepackage{graphicx}
% for neatly defining theorems and propositions
%\usepackage{amsthm}
% making logically defined graphics
%%%\usepackage{xypic}

% there are many more packages, add them here as you need them

% define commands here
\begin{document}
If a given function $f$ satisfies a \PMlinkescapetext{simple} differential equation, the Taylor series \PMlinkescapetext{expansion} of $f$ can sometimes be obtained easily.

Let 
               $$f(x) = \sin(m\arcsin x),$$
where $m$ is a non-zero \PMlinkescapetext{constant}, be an example (\PMlinkname{cf.}{Cf} the cyclometric functions).\, We form the derivatives
          $$f'(x) = \frac{m}{\sqrt{1-x^2}}\cos(m\arcsin x),$$
            $$f''(x) = -\frac{m^2}{1-x^2}\sin(m\arcsin x)
                  +\frac{mx}{(1-x^2)\sqrt{1-x^2}}\cos(m\arcsin x),$$
which show that $f$ satisfies the differential equation
            $$(1-x^2)f''-xf'+m^2f = 0.$$
Differentiating this repeatedly gives the equations
            $$(1-x^2)f'''-3xf''+(m^2-1)f' = 0,$$
            $$(1-x^2)f^{(4)}-5xf'''+(m^2-4)f'' = 0,$$
and so on.\, Using the sum of odd numbers\, $1+3+5+\cdots+(2n\!-\!1) = n^2$\, and induction on $n$ yields the recurrence relation
    $$(1-x^2)f^{(n+2)}-(2n+1)xf^{(n+1)}+(m^2-n^2)f^{(n)} = 0.$$
Plugging in \, $x = 0$\, yields
     $$f^{(n+2)}(0) = (n^2-m^2)f^{(n)}(0) \quad (n = 0,\,1,\,2,\,...).$$
Since\, $f'(0) = m$,\, we have that
     $$f^{(2n+1)}(0) = m(1^2-m^2)(3^2-m^2)\ldots((2n-1)^2-m^2),$$
whereas all even derivatives of $f$ vanish at $x=0$.\, (Note that $f$ is an odd function.)\, Thus, we obtain the Taylor \PMlinkescapetext{expansion} of $f$:
$$\sin(m\arcsin x) = \frac{m}{1!}x+\frac{m(1^2-m^2)}{3!}x^3+
\frac{m(1^2-m^2)(3^2-m^2)}{5!}x^5+\cdots$$
By the ratio test, this series converges for\, $|x| < 1$.

\begin{thebibliography}{8}
\bibitem{lindelof}{\sc Ernst Lindel\"of:} {\em Differentiali- ja integralilasku 
ja sen sovellutukset} I. \,WSOY. Helsinki (1950).
\end{thebibliography}
%%%%%
%%%%%
\end{document}
