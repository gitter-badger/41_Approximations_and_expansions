\documentclass[12pt]{article}
\usepackage{pmmeta}
\pmcanonicalname{LectureNotesOnPolynomialInterpolation}
\pmcreated{2013-03-22 16:51:59}
\pmmodified{2013-03-22 16:51:59}
\pmowner{rmilson}{146}
\pmmodifier{rmilson}{146}
\pmtitle{lecture notes on polynomial interpolation}
\pmrecord{7}{39116}
\pmprivacy{1}
\pmauthor{rmilson}{146}
\pmtype{Topic}
\pmcomment{trigger rebuild}
\pmclassification{msc}{41A05}
\pmrelated{EvaluationHomomorphism}
\pmrelated{LagrangeInterpolationFormula}
\pmrelated{VanderMondeMatrix}
\pmdefines{multi-evaluation mapping}
\pmdefines{interpolation mapping}
\pmdefines{polynomial interpolation}

\endmetadata

\usepackage{amsmath}
\usepackage{amsfonts}
\usepackage{amssymb}
\usepackage{amsthm}

\newtheorem{proposition}{Proposition}
\newtheorem{theorem}[proposition]{Theorem}
\theoremstyle{definition}
\newtheorem{definition}[proposition]{Definition}
\newtheorem{remark}[proposition]{Remark}
\newtheorem{example}[proposition]{Example}
\newtheorem{problem}[proposition]{Problem}

\newcommand{\bmat}[1]{\begin{bmatrix}#1\end{bmatrix}}


\newcommand{\be}{\mathbf{e}}
\newcommand{\bx}{\mathbf{x}}
\newcommand{\bu}{\mathbf{u}}
\newcommand{\bv}{\mathbf{v}}

\newcommand{\by}{\mathbf{y}}
\newcommand{\ev}{\operatorname{ev}}
\newcommand{\pol}{\operatorname{pol}}
\newcommand{\img}{\operatorname{Im}}
\newcommand{\orank}{\operatorname{rank}}

\newcommand{\cP}{\mathcal{P}}
\newcommand{\cU}{\mathcal{U}}
\newcommand{\cV}{\mathcal{V}}
\newcommand{\VM}{\mathsf{VM}}
\newcommand{\VD}{\mathsf{V}}
\newcommand{\Rset}{\mathbb{R}}


\begin{document}
\section{Summary of notation and terminonlogy}
\begin{itemize}
\item \emph{multi-evaluation mapping}: $\ev_{\bx}: \cP_m \to \Rset^n,\; 
  \bx\in\Rset^n$.  Here $\cP_m$ is the vector space of polynomials of
  degree $m$ or less. 
\item \emph{interpolation mapping}: $\pol_{\bx}: \Rset^{n} \to
  \cP_{n-1},\; \bx\in \Rset^{n}$ where $x_1,\ldots,x_{n}$ are
  distinct;
\item \emph{Vandermonde matrix and polynomial};
\item \emph{overdetermined and underdetermined linear system;}
\item \emph{overdetermined and underdetermined interpolation problem}
\end{itemize}

\section{The polynomial interpolation problem}
\begin{definition}
  Let $\bx\in \Rset^n$ be given.  Define $\ev_{\bx}:\cP_m\to
  \Rset^n$, the multi-evaluation mapping, to be the linear
  transformation given by
  \[ \ev_{\bx} : p \to \bmat{p(x_1)\\ \vdots \\ p(x_n)},\quad
  p\in \cP_m.\]
\end{definition}

\begin{problem}[Polynomial interpolation]
Given $(x_1,y_1),\ldots, (x_n,y_n)\in \Rset^2$
to find a polynomial $p\in \cP_m$ such that $p(x_i)=y_i,\;
i=1,\ldots, n$.  Equivalently, given $\bx,\by\in \Rset^n$ to find the
preimage $\ev_{\bx}^{-1}(\by)$.
\end{problem}

\begin{theorem}
  Fix $\bx\in \Rset^{n}$.  The multi-evaluation mapping
  $\ev_{\bx}:\cP_{n-1}\to \Rset^{n}$  is an isomorphism if and only
  if the components $x_1,\ldots, x_n$ are distinct.
\end{theorem}


\section{Vandermonde matrix, polynomial, and determinant}
\begin{definition}
  For a given $\bx\in \Rset^{n}$, the following matrix
  \[ \VM(\bx) = \VM(x_1,\ldots, x_{n}) = 
  \bmat{
    1 & x_1 & x_1^{2} &\ldots & x_1^{n-1} \\
    1 & x_2 & x_2^{2} &\ldots &x_2^{n-1} \\
    1 & x_{3} & x_{3}^{2} & \ldots & x_{3}^{n-1} \\
    \vdots & \vdots & \vdots & \ddots  & \vdots \\
    1 & x_{n} & x_{n}^{2} & \ldots & x_{n}^{n-1} }
  \] 
  is called
  the \emph{Vandermonde matrix}.  The
  expression,
  \[ \VD(\bx) = \VD(x_1,\ldots, x_{n}) = \prod_{1\leq i< j\leq n}
    (x_j-x_i)\] 
    is called the Vandermonde  polynomial.  Note: we define $\VD(x_1)
    =1$; the usual convention is that the empty product is equal to $1$.
\end{definition}

\begin{proposition}
  The Vandermonde matrix is the transformation matrix of $\ev_{\bx}$ with the
  monomial basis $[1,x,\ldots, x^n]$ as the input basis and the
  standard basis $[\be_1,\ldots,\be_{n+1}]$ as the output basis.
\end{proposition}

\begin{theorem}
  The Vandermonde polynomial gives the determinant of the Vandermonde
  matrix:
  \begin{equation}
    \label{eq:vmid}
    \left| \begin{matrix}
        1 & x_1 & \ldots & x_1^{n-1} \\
        1 & x_2 & \ldots &x_2^{n-1} \\
        \vdots &  \vdots & \ddots  & \vdots \\
        1 & x_{n}  & \ldots & x_{n}^{n-1} 
      \end{matrix}
    \right| = \prod_{1\leq i< j\leq n+1}
    (x_j-x_i),    
  \end{equation}
    or more succinctly,
    \[ \det \VM(\bx) = \VD(\bx).\]
\end{theorem}
\begin{proof}
  We will prove formula \eqref{eq:vmid} by induction on $n$.  Note
  that
  \[ \VM(x_1) = \bmat{1},\quad \VD(x_1) = 1 .\] Evidently then, the
  formula works for $n=1$. Next, suppose that we believe the formula
  for a given $n$.  We show that the formula is valid for $n+1$. For
  $x_1,\ldots, x_{n}\in \Rset$ and a variable $x$, and consider the
  $n^{\text{th}}$ degree polynomial
  \[ p(x)=\det \VM(x_1,\ldots, x_n,x) = \left| \begin{matrix}
        1 & x_1 & \ldots & x_1^{n-1}& x_1^{n} \\
        1 & x_2 & \ldots &x_2^{n-1}&x_2^{n} \\
        \vdots & \vdots & \ddots  &\vdots& \vdots \\
        1& x_n & \ldots & x_n^{n-1} & x_n^n\\
        1 & x &  \ldots & x^{n-1}& x^n \end{matrix}
    \right|\]
    By the properties of determinants, $x_1,\ldots, x_n$ are roots of
    $p(x)$.  Taking the cofactor expansion along the bottom row, we
    see that the coefficient of $x^n$ is $\VD(x_1,\ldots, x_n)$.
    Therefore,
    \[ p(x) = \VD(x_1,\ldots, x_n) (x-x_1) \cdots (x-x_n) =
    \VD(x_1,\ldots, x_n, x),\]
    as was to be shown.
\end{proof}
\section{Lagrange interpolation formula}  Let
$x_1,\ldots, x_n\in \Rset$ be distinct.  We know that
$\ev_\bx:\cP_{n-1}\to\Rset^n$ is invertible.  Let
$\pol_\bx:\Rset^n\to \cP_{n-1}$ denote the inverse.  In principle,
this inverse is described by the inverse of the Vandermonde matrix.
Is there another way to solve the interpolation problem?  For
$i=1,\ldots, n$ let us define the polynomial
\[ p_i(x) = \prod_{j\neq i} \frac{x-x_j}{x_i- x_j} .\] These $n-1$
degree polynomials have been ``engineered'' so that $p_i(x_i) = 1$ and
so that $p_i(x_j) = 0$ for $i\neq j$
\begin{theorem}[Lagrange interpolation formula]
  Let $x_1,\ldots, x_n\in \Rset$ be distinct.  Then,
  \[ \pol_\bx(\by) = y_1 p_1 + \cdots + y_n p_n.\]  
\end{theorem}


\section{Underdetermined and overdetermined interpolation}
\begin{definition}
  Let $T:\cU\to \cV$ be a linear transformation of finite dimensional
  vector spaces.  A linear problem is an equation of the form
  \[ T(\bu) = \bv, \] where $\bv\in \cV$ is given, and $\bu\in \cU$ is
  the unknown.  To be more precise, the problem is to determine
  the preimage $T^{-1}(\bv)$ for a given $\bv\in \cV$.  We say that the
  problem is overdetermined if $\dim\cV>\dim \cU$, i.e. if there
  are more equations than unknowns.  The linear problem is said to be
  underdetermined if $\dim\cV<\dim\cU$, i.e., if there are more
  variables than equations.
\end{definition}

\begin{remark}
By the rank-nullity theorem, an
underdetermined linear problem will either be inconsistent, or will
have multiple solutions.  Thus, an underdetermined system arises when
$T$ is not one-to-one; i.e., the kernel $\ker(T)$ is non-trivial.  An
overdetermined linear system is inconsistent, unless $\bv$ satisfies a
number linear compatibility constraints, equations that describe the image
$\img(T)$.  To put it another way, an overdetermined system arises
when $T$ is not onto.
\end{remark}

\begin{definition} Let $x_1,\ldots, x_n\in \Rset$ be distinct and let
  $\ev_\bx:\cP_m\to \Rset^n$ be the corresponding multi-evaluation
  mapping.  Let $\by\in \Rset^n$ be given.  The linear equation
  \[ \ev_{\bx}(p) = \by,\quad p \in \cP_m\]
  is called an underdetermined interpolation problem if $m\geq n$.
  If $m\leq n-2$, we call the above equation an overdetermined
  interpolation problem.  Note that if $m=n-1$, the interpolation
  problem is ``just right''; there is exactly one solution, namely the
  polynomial
  $\pol_\bx(\by)$ given by the Lagrange interpolation formula.
\end{definition}
\begin{proposition}[Underdetermined interpolation]
  Let $x_1,\ldots, x_n\in \Rset$ be distinct.  Define
  \[ q(x) = (x-x_1)\cdots (x-x_n)\]
  to be the $n^{\text{th}}$ degree polynomial with the $x_i$ as its roots.
  Suppose that $m\geq n$. Then,
  the multi-evaluation mapping
  $\ev_\bx:\cP_m\to \Rset^n$ is not one-to-one.  A basis for the
  kernel is given by  $q(x), x q(x),\ldots,
  x^{m-n} q(x)$.  Let $y_1,\ldots, y_n\in \Rset$ be given. The solution
  set to the interpolation problem $\ev_\bx(p) = \by$, is given by
  \[ \ev_\bx^{-1}(\by) = \{ r + s q : s \in \cP_{m-n} \}, \]
  where
  $r = \pol_\bx(\by)$
  is the $n-1^\text{st}$ degree polynomial given by the Lagrange
  interpolation formula.  To put it another way, the general solution
  of the equation
  \[ \ev_\bx(p)= \by \]
  is given by
  \[ p = r + s q,\quad s\in \cP_{n-m}\text{ free.}\]
\end{proposition}
\begin{proposition}[Overdetermined interpolation]
   Let $x_1,\ldots, x_n\in \Rset$ be distinct. 
  Suppose that $m\leq n-2$. Then,
  the multi-evaluation mapping
  $\ev_\bx:\cP_m\to \Rset^n$ is not onto.  If $m=n-2$, then
  $\by\in\Rset^4$ belongs to $\img(\ev_\bx)$ if and only  if
  \[ \left| \begin{matrix}
        1 & x_1 & \ldots & x_1^{m}& y_1 \\
        1 & x_2 & \ldots &x_2^{m}&y_2 \\
        \vdots & \vdots & \ddots  & \vdots &\vdots\\
        1 & x_{n-1} & \ldots &x_{n-1}^{m}&y_{n-1} \\
        1& x_n & \ldots & x_n^{m} & y_n
      \end{matrix}
    \right|=0\]
    More generally, for any $m\leq n-2$, the interpolation problem
    $\ev_\bx(p)=\by,\; \by\in \Rset^n$ has a solution if and only
    if $\orank M(\bx,\by) = m+1$ where
    \[ M(\bx,\by)= \begin{bmatrix}
        1 & x_1 & \ldots & x_1^{m}& y_1 \\
        1 & x_2 & \ldots &x_2^{m}&y_2 \\
        \vdots & \vdots & \ddots  & \vdots &\vdots\\
        1 & x_{m+1} & \ldots &x_{m+1}^{m}&y_{m+1} \\
        \vdots & \vdots & \vdots & \vdots &\vdots\\
        1& x_n & \ldots & x_n^{m} & y_n
      \end{bmatrix} \]
\end{proposition}
\begin{remark}
The compatibility constraints on
$y_1,\ldots, y_n$ for an overdetermined system amount to the condition
that $\by$ lie in the image of $\ev_{\bx}$, or what is equivalent,
belong to the column space of the corresponding transformation matrix.
We can therefore obtain the constraint equations by row reducing the
augmented matrix $M(\bx,\by)$ to echelon form.  The back entries of
the last $n-m-1$ rows will hold the constraint equations.
Equivalently, we can solve the interpolation problem for
$(x_1,y_1),\ldots, (x_m,y_m)$ to obtain a $p=y_1 p_1 + y_{m+1}
p_{m+1}\in \cP_m$.  The additional equations
\[ y_i = y_1 p_1(x_i)+\cdots + y_{m+1} p_{m+1}(x_i),\; i=m+2,\ldots,
n\] are the desired compatibility constraints on $y_1,\ldots, y_n$.
  
\end{remark}

%%%%%
%%%%%
\end{document}
