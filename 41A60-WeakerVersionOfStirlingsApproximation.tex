\documentclass[12pt]{article}
\usepackage{pmmeta}
\pmcanonicalname{WeakerVersionOfStirlingsApproximation}
\pmcreated{2013-03-22 16:25:21}
\pmmodified{2013-03-22 16:25:21}
\pmowner{rm50}{10146}
\pmmodifier{rm50}{10146}
\pmtitle{weaker version of Stirling's approximation}
\pmrecord{7}{38573}
\pmprivacy{1}
\pmauthor{rm50}{10146}
\pmtype{Result}
\pmcomment{trigger rebuild}
\pmclassification{msc}{41A60}
\pmclassification{msc}{30E15}
\pmclassification{msc}{68Q25}

% this is the default PlanetMath preamble.  as your knowledge
% of TeX increases, you will probably want to edit this, but
% it should be fine as is for beginners.

% almost certainly you want these
\usepackage{amssymb}
\usepackage{amsmath}
\usepackage{amsfonts}

% used for TeXing text within eps files
%\usepackage{psfrag}
% need this for including graphics (\includegraphics)
%\usepackage{graphicx}
% for neatly defining theorems and propositions
%\usepackage{amsthm}
% making logically defined graphics
%%%\usepackage{xypic}

% there are many more packages, add them here as you need them

% define commands here

\begin{document}
One can prove a weaker version of Stirling's approximation without appealing to the gamma function. Consider the graph of $\ln x$ and note that
\[\ln(n-1)!\leq \int_1^n \ln x \,\mathrm{d} x\leq \ln n!\]
But $\int \ln x \,\mathrm{d} x=x\ln x-x$, so
\[\ln(n-1)!\leq n\ln n-n+1\leq \ln n!\]
and thus
\[n\ln n-n+1+\ln n\geq\ln(n-1)!+\ln n=\ln n!\geq n\ln n-n+1\]
so
\[\ln n-1+\frac{1}{n}+\frac{\ln n}{n}\geq\frac{1}{n}\ln n!\geq\ln n-1+\frac{1}{n}\]
As $n$ gets large, the expressions on either end approach $\ln n-1$, so we have

\[\frac{1}{n}\ln n! \approx \ln n - 1\]

Multiplying through by $n$ and exponentiating, we get
\[n!\approx n^ne^{-n}\]
%%%%%
%%%%%
\end{document}
