\documentclass[12pt]{article}
\usepackage{pmmeta}
\pmcanonicalname{Simpsons38Rule}
\pmcreated{2013-03-22 13:40:56}
\pmmodified{2013-03-22 13:40:56}
\pmowner{Daume}{40}
\pmmodifier{Daume}{40}
\pmtitle{Simpson's 3/8 rule}
\pmrecord{11}{34351}
\pmprivacy{1}
\pmauthor{Daume}{40}
\pmtype{Definition}
\pmcomment{trigger rebuild}
\pmclassification{msc}{41A05}
\pmclassification{msc}{41A55}

\endmetadata

% this is the default PlanetMath preamble.  as your knowledge
% of TeX increases, you will probably want to edit this, but
% it should be fine as is for beginners.

% almost certainly you want these
\usepackage{amssymb}
\usepackage{amsmath}
\usepackage{amsfonts}

% used for TeXing text within eps files
%\usepackage{psfrag}
% need this for including graphics (\includegraphics)
%\usepackage{graphicx}
% for neatly defining theorems and propositions
%\usepackage{amsthm}
% making logically defined graphics
%%%\usepackage{xypic}

% there are many more packages, add them here as you need them

% define commands here
\begin{document}
Simpson's $\frac{3}{8}$ rule is a method for approximating a
definite integral by evaluating the integrand at finitely many
points.  The formal rule is given by
\[
\int_{x_{0}}^{x_{3}}f(x)\,dx\;\approx\;\frac{3h}{8}\left[f(x_{0})+3f(x_{1})+3f(x_{2})+f(x_{3})\right]
\]
where $x_1=x_0+h$, $x_2=x_0+2h$, $x_3=x_0+3h$.



Simpson's $\frac{3}{8}$ rule is the third Newton-Cotes quadrature
formula. It has degree of precision 3. This means it is exact for
polynomials of degree less than or equal to three. Simpson's
$\frac{3}{8}$ rule is an improvement to the traditional Simpson's
rule. The extra function evaluation gives a slightly more accurate
approximation . We can see this with an example.





Using the fundamental theorem of the calculus, one shows

\[
\int_{0}^{\pi}\sin(x)\,dx =2.
\]

In this case Simpson's rule gives,
\[
\int_{0}^{\pi}\sin(x)\,dx\,\approx\;\frac{\pi}{6}\left[\sin(0)+4\sin\left(\frac{\pi}{2}\right)+\sin(\pi)\right]\,=\,2.094
\]


However, Simpson's $\frac{3}{8}$ rule does slightly better.

\[
\int_{0}^{\pi}\sin(x)\,dx\,\approx\;\left(\frac{3}{8}\right)\frac{\pi}{3}\left[\sin(0)+3\sin\left(\frac{\pi}{3}\right)+3\sin\left(\frac{2\pi}{3}\right)+\sin(\pi)\right]\,=\,2.040
\]
%%%%%
%%%%%
\end{document}
