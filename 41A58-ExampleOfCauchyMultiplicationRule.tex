\documentclass[12pt]{article}
\usepackage{pmmeta}
\pmcanonicalname{ExampleOfCauchyMultiplicationRule}
\pmcreated{2013-03-22 17:29:19}
\pmmodified{2013-03-22 17:29:19}
\pmowner{pahio}{2872}
\pmmodifier{pahio}{2872}
\pmtitle{example of Cauchy multiplication rule}
\pmrecord{8}{39876}
\pmprivacy{1}
\pmauthor{pahio}{2872}
\pmtype{Example}
\pmcomment{trigger rebuild}
\pmclassification{msc}{41A58}
\pmclassification{msc}{40-00}
\pmclassification{msc}{30B10}
\pmclassification{msc}{26A24}
\pmsynonym{Taylor series gotten by multiplication}{ExampleOfCauchyMultiplicationRule}
%\pmkeywords{Taylor series of function of two variables}
\pmrelated{HyperbolicFunctions}

% this is the default PlanetMath preamble.  as your knowledge
% of TeX increases, you will probably want to edit this, but
% it should be fine as is for beginners.

% almost certainly you want these
\usepackage{amssymb}
\usepackage{amsmath}
\usepackage{amsfonts}

% used for TeXing text within eps files
%\usepackage{psfrag}
% need this for including graphics (\includegraphics)
%\usepackage{graphicx}
% for neatly defining theorems and propositions
 \usepackage{amsthm}
% making logically defined graphics
%%%\usepackage{xypic}

% there are many more packages, add them here as you need them

% define commands here

\theoremstyle{definition}
\newtheorem*{thmplain}{Theorem}

\begin{document}
\PMlinkescapeword{terms}

Let us form the Taylor expansion of\, $e^x\sin{y}$\, starting from the known Taylor expansions
$$e^x = 1+x+\frac{x^2}{2!}+\frac{x^3}{3!}+\ldots,$$
$$\sin{y} = y-\frac{y^3}{3!}+\frac{y^5}{5!}-\frac{y^7}{7!}+-\ldots$$
and multiplying these series with Cauchy multiplication rule.  As power series, both series are absolutely convergent for all real (and complex) values of $x$ and $y$.  The rule gives immediately the series
\begin{align} 
y\!+\!(-\frac{y^3}{3!}\!+\!xy)\!+\!(\frac{y^5}{5!}\!-\!\frac{xy^3}{3!}\!+\!\frac{x^2y}{2!})\!
+\!(-\frac{y^7}{7!}\!+\!\frac{xy^5}{5!}\!-\!\frac{x^2y^3}{2!3!}\!+\!\frac{x^3y}{3!})\!
+\!(\frac{y^9}{9!}\!-\!\frac{xy^7}{7!}\!+\!\frac{x^2y^5}{2!5!}\!-\!\frac{x^3y^3}{3!3!}\!+\!\frac{x^4y}{4!})\!+\!...
\end{align}
The parenthesis expressions here seem a bit irregular, but we can regroup and rearrange the terms in new parentheses:
\begin{align}
e^x\sin{y} \,=\, 
y+\frac{xy}{1!1!}+\left(\frac{x^2y}{2!1!}-\frac{y^3}{3!}\right)+\left(\frac{x^3y}{3!1!}-\frac{xy^3}{1!3!}\right)
+\left(\frac{x^4y}{4!1!}-\frac{x^2y^3}{2!3!}+\frac{y^5}{5!}\right)+\ldots
\end{align}
It's clear that the last series \PMlinkescapetext{contains} precisely the same terms as the preceding one.  The regrouping and the rearranging of the terms is allowable, since also (1) is converges absolutely.  In fact, if one would multiply the series of $e^x$ with the series $y\!+\!\frac{y^3}{3!}\!+\!\frac{y^5}{5!}\!+\!...$ of $\sinh{y}$ (which converges absolutely $\forall x\in\mathbb{R}$), one would get the series like (1) but all signs ``+''; by the Cauchy multiplication rule this series converges especially for each positive $x$ and $y$, in which case it is a series with positive terms; hence (1) is absolutely convergent.

The form (2) can be obtained directly from the \PMlinkname{Taylor series formula}{TaylorSeries}.
%%%%%
%%%%%
\end{document}
